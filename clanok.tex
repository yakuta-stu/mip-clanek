\documentclass[10pt,twoside,czech,a4paper]{article}

\usepackage[czech]{babel}
\usepackage[utf8]{inputenc}
\usepackage{graphicx}
\usepackage{url}
\usepackage{hyperref}

\usepackage{cite}

\pagestyle{headings}

\title{P2P Protokoly pro přenos souborů\thanks{Semestrální projekt v předmětu Metody inženýrské práce, ak. rok 2023/24, vedení: Ing. Richard Marko, PhD.}}

\author{Andrei Yakuta\\[2pt]
	{\small Slovenská technická univerzita v Bratislave}\\
	{\small Fakulta informatiky a informačných technológií}\\
	{\small \texttt{xyakuta@stuba.sk}}
	}

\date{\small 12. října 2023}



\begin{document}

\maketitle

\begin{abstract}
\ldots
\end{abstract}



\section{Úvod}

P2P (Peer-to-Peer) protokoly pro přenos souborů představují způsob sdílení dat mezi uzly ve stejné síti bez nutnosti centrálního serveru.
Tento způsob sdílení dat umožňuje výrazné snížení nákladů na infrastrukturu, zatímco zajišťuje rychlý a spolehlivý přenos velkých souborů.
V mém projektu se zaměřím na analýzu různých P2P protokolů, jako jsou BitTorrent, eDonkey a Gnutella.
Hlavním cílem je porozumění architektuře těchto protokolů, mechanismům vyhledávání a sdílení souborů, a způsobům, jak tyto protokoly řeší problémy s bezpečností a soukromím.
Porovnám také P2P protokoly s tradičními client-server modely, abych diskutoval o výhodách a nevýhodách obou přístupů.
Důležitou součástí mé analýzy bude hodnocení různých implementací P2P protokolů a jejich dopad na výkon, škálovatelnost a odolnost proti chybám.
Dále se pokusím předvídat možný vývoj P2P technologií a zvážit jejich potenciální dopad na budoucí aplikace a služby v digitálním světě.


\bibliography{literatura}
\bibliographystyle{plain} % prípadne alpha, abbrv alebo hociktorý iný
\end{document}
